%%%%%%%%%%%%%%%%%%%%%%%%%%%%%%%%%%%%%%%%%%
%Copyright (C) 2018-2020  YuZJLab
%This program is free software: you can redistribute it and/or modify
%it under the terms of the GNU General Public License as published by
%the Free Software Foundation, either version 3 of the License, or
%(at your option) any later version.
%This program is distributed in the hope that it will be useful,
%but WITHOUT ANY WARRANTY; without even the implied warranty of
%MERCHANTABILITY or FITNESS FOR A PARTICULAR PURPOSE.  See the
%GNU General Public License for more details.
%You should have received a copy of the GNU General Public License
%along with this program.  If not, see <https://www.gnu.org/licenses/>.
%%%%%%%%%%%%%%%%%%%%%%%%%%%%%%%%%%%%%%%%%%
\chapter{Distribution:分发}
GNU Emacs is free software; this means that everyone is free to use it and free to redistribute it under certain conditions. GNU Emacs is not in the public domain; it is copyrighted and there are restrictions on its distribution, but these restrictions are designed to permit everything that a good cooperating citizen would want to do. What is not allowed is to try to prevent others from further sharing any version of GNU Emacs that they might get from you. The precise conditions are found in the GNU General Public License that comes with Emacs and also appears in this manual\footnote{This manual is itself covered by the GNU Free Documentation License. This license is similar in spirit to the General Public License, but is more suitable for documentation. See Appendix B [GNU Free Documentation License]}. See Appendix A [Copying], page 495.\par
GNU Emacs 是自由软件;这意味着所有人都能在不违反相关条例的情况下自由地使用和重新构建。GNU Emacs并不被放在公共区域\footnote{译注:也就是说,你不能为所欲为。};它是受到版权保护并且其分发是有限制的。但是这些限制的目的是保证一个品行端正且愿意合作的公民做其所希望做的事情的自由。试图防止他人分享可能从你那里得到的任何GNU Emacs版本是被禁止的。精确的条例将可以在随附GNU Emacs提供的GNU General Public Licence或者这份手册\footnote{这份手册本身是受GNU Free Documentation License保护的:它的精神接近GNU General Public License,但更适合用于文档。参见\ref{chap:FDL}【GNU Free Documentation License】,第\pageref{chap:FDL}页}的底部:参见\ref{chap:gpl}【GNU General Public License】,第\pageref{chap:gpl}页。\par
One way to get a copy of GNU Emacs is from someone else who has it. You need not ask for our permission to do so, or tell anyone else; just copy it. If you have access to the Internet, you can get the latest distribution version of GNU Emacs by anonymous FTP; see \url{https://www.gnu.org/software/emacs} on our website for more information.\par
一种得到GNU Emacs副本的方法是从其他拥有副本的人那里获取。你不需要为此事征求许可或告诉任何其他人;你只需要复制它即可。如果你能连接到互联网,你可以从匿名的FTP服务器\footnote{译注:FTP是“文件共享协议”的缩写,你能够从搭建FTP服务器的站点下载文件。}中得到GNU Emacs的最新副本;参见\url{https://www.gnu.org/software/emacs}以得到更多信息。\par
You may also receive GNU Emacs when you buy a computer. Computer manufacturers are free to distribute copies on the same terms that apply to everyone else. These terms require them to give you the full sources, including whatever changes they may have made, and to permit you to redistribute the GNU Emacs received from them under the usual terms of the General Public License. In other words, the program must be free for you when you get it, not just free for the manufacturer.\par
你也可能在你购买电脑时获得一份GNU Emacs副本\footnote{译注:从目前的主流制造商来看,这似乎不太可能。}。计算机制造商可以按照适用于其他人的条例自由地分发GNU Emacs。这些条款要求他们给出包含他们所做出的改变的完整的源代码,并且允许你按照GNU General Public License重建从他们那里获得的GNU Emacs。换言之,这个程序必须在你获得它是是自由的,而不是仅对生产商自由。\par
If you find GNU Emacs useful, please send a donation to the Free Software Foundation to support our work. Donations to the Free Software Foundation are tax-deductible in the US. If you use GNU Emacs at your workplace, please suggest that the company make a donation. To donate, see \url{https://my.fsf.org/donate/}. For other ways in which you can help, see \url{https://www.gnu.org/help/help.html}.\par
如果你觉得GNU Emacs有用,请捐款给自由软件基金会(Free Software Foundation)以支持我们。对于自由软件基金会的捐赠在美国是免税的。如果你在你的工作处使用GNU Emacs,请建议你的公司作出捐赠。如果你对于捐赠感兴趣,请看\url{https://my.fsf.org/donate/}。你也可以参照\url{https://www.gnu.org/help/help.html}如果你希望以其他形式提供帮助。\par
We also sell hardcopy versions of this manual and \textit{An Introduction to Programming in Emacs Lisp}, by Robert J. Chassell. You can visit our online store at \url{https://shop.fsf.org/}. The income from sales goes to support the foundation’s purpose: the development of new free software, and improvements to our existing programs including GNU Emacs.\par
我们也出售这份手册的纸质版本\footnote{译注:当然,这里指的是没有中文翻译的原版。}以及由Robert J. Chassell编著的《An Introduction to Programming in Emacs Lisp》。你可以在\url{https://shop.fsf.org/}浏览我们的在线商店,它的收入被用于支持基金会的意图:新软件的开发以及旧软件(包括GNU Emacs)的提高。\par
If you need to contact the Free Software Foundation, see\url{ https://www.fsf.org/about/contact/}, or write to\par
如果你希望与自由软件基金会联系,参见\url{ https://www.fsf.org/about/contact/}或者写信给:\par
Free Software Foundation\par
51 Franklin Street, Fifth Floor\par
Boston, MA 02110-1301\par
USA
\section{英文版致谢}
Contributors to GNU Emacs include (为GNU Emacs提供贡献的包括) Jari Aalto, Per Abrahamsen, Tomas Abrahamsson, Jay K. Adams, Alon Albert, Michael Albinus, Nagy Andras, Benjamin Andresen, Ralf Angeli, Dmitry Antipov, Joe Arceneaux, Emil ˚ Astr¨om, Miles Bader, David Bakhash, Juanma Barranquero, Eli Barzilay, Thomas Baumann, Steven L. Baur, Jay Belanger, Alexander L. Belikoff, Thomas Bellman, Scott Bender, Boaz Ben-Zvi, Sergey Berezin, Stephen Berman, Karl Berry, Anna M. Bigatti, Ray Blaak, Martin Blais, Jim Blandy, Johan Bockg˚ard, Jan B¨ocker, Joel Boehland, Lennart Borgman, Per Bothner, Terrence Brannon, Frank Bresz, Peter Breton, Emmanuel Briot, Kevin Broadey, Vincent Broman, Michael Brouwer, David M. Brown, Ken Brown, Stefan Bruda, Georges Brun-Cottan, Joe Buehler, Scott Byer, W lodek Bzyl, Tino Calancha, Bill Carpenter, Per Cederqvist, Hans Chalupsky, Chris Chase, Bob Chassell, Andrew Choi, Chong Yidong, Sacha Chua, Stewart Clamen, James Clark, Mike Clarkson, Glynn Clements, Andrew Cohen, Daniel Colascione, Christoph Conrad, Ludovic Court`es, Andrew Csillag, Toby Cubitt, Baoqiu Cui, Doug Cutting, Mathias Dahl, Julien Danjou, Satyaki Das, Vivek Dasmohapatra, Dan Davison, Michael DeCorte, Gary Delp, Nachum Dershowitz, Dave Detlefs, Matthieu Devin, Christophe de Dinechin, Eri Ding, Jan Dj¨arv, Lawrence R. Dodd, Carsten Dominik, Scott Draves, Benjamin Drieu, Viktor Dukhovni, Jacques Duthen, Dmitry Dzhus, John Eaton, Rolf Ebert, Carl Edman, David Edmondson, Paul Eggert, Stephen Eglen, Christian Egli, Torbj¨orn Einarsson, Tsugutomo Enami, David Engster, Hans Henrik Eriksen, Michael Ernst, Ata Etemadi, Frederick Farnbach, Oscar Figueiredo, Fred Fish, Steve Fisk, Karl Fogel, Gary Foster, Eric S. Fraga, Romain Francoise, Noah Friedman, Andreas Fuchs, Shigeru Fukaya, Xue Fuqiao, Hallvard Furuseth, Keith Gabryelski, Peter S. Galbraith, Kevin Gallagher, Fabi´an E. Gallina, Kevin Gallo, Juan Le´on Lahoz Garc´ıa, Howard Gayle, Daniel German, Stephen Gildea, Julien Gilles, David Gillespie, Bob Glickstein, Deepak Goel, David De La Harpe Golden, Boris Goldowsky, David Goodger, Chris Gray, Kevin Greiner, Michelangelo Grigni, Odd Gripenstam, Kai Großjohann, Michael Gschwind, Bastien Guerry, Henry Guillaume, Dmitry Gutov, Doug Gwyn, Bruno Haible, Ken’ichi Handa, Lars Hansen, Chris Hanson, Jesper Harder, Alexandru Harsanyi, K. Shane Hartman, John Heidemann, Jon K. Hellan, Magnus Henoch, Markus Heritsch, Dirk Herrmann, Karl Heuer, Manabu Higashida, Konrad Hinsen, Anders Holst, Jeffrey C. Honig, Tassilo Horn, Kurt Hornik, Tom Houlder, Joakim Hove, Denis Howe, Lars Ingebrigtsen, Andrew Innes, Seiichiro Inoue, Philip Jackson, Martyn Jago, Pavel Janik, Paul Jarc, Ulf Jasper, Thorsten Jolitz, Michael K. Johnson, Kyle Jones, Terry Jones, Simon Josefsson, Alexandre Julliard, Arne Jørgensen, Tomoji Kagatani, Brewster Kahle, Tokuya Kameshima, Lute Kamstra, Ivan Kanis, David Kastrup, David Kaufman, Henry Kautz, Taichi Kawabata, Taro Kawagishi, Howard Kaye, Michael Kifer, Richard King, Peter Kleiweg, Karel Kl´ıˇc, Shuhei Kobayashi, Pavel Kobyakov, Larry K. Kolodney, David M. Koppelman, Koseki Yoshinori, Robert Krawitz, Sebastian Kremer, Ryszard Kubiak, Igor Kuzmin, David K˚agedal, Daniel LaLiberte, Karl Landstrom, Mario Lang, Aaron Larson, James R. Larus, Vinicius Jose Latorre, Werner Lemberg, Frederic Lepied, Peter Liljenberg, Christian Limpach, Lars Lindberg, Chris Lindblad, Anders Lindgren, Thomas Link, Juri Linkov, Francis Litterio, Sergey Litvinov, Leo Liu, Emilio C. Lopes, Martin Lorentzon, Dave Love, Eric Ludlam, K´aroly L˝orentey, Sascha L¨udecke, Greg McGary, Roland McGrath, Michael McNamara, Alan Mackenzie, Christopher J. Madsen, Neil M. Mager, Artur Malabarba, Ken Manheimer, Bill Mann, Brian Marick, Simon Marshall, Bengt Martensson, Charlie Martin, Yukihiro Matsumoto, Tomohiro Matsuyama, David Maus, Thomas May, Will Mengarini, David Megginson, Stefan Merten, Ben A. Mesander, Wayne Mesard, Brad Miller, Lawrence Mitchell, Richard Mlynarik, Gerd M¨ollmann, Dani Moncayo, Stefan Monnier, Keith Moore, Jan Moringen, Morioka Tomohiko, Glenn Morris, Don Morrison, Diane Murray, Riccardo Murri, Sen Nagata, Erik Naggum, Gergely Nagy, Nobuyoshi Nakada, Thomas Neumann, Mike Newton, Thien-Thi Nguyen, Jurgen Nickelsen, Dan Nicolaescu, Hrvoje Nikˇsi´c, Jeff Norden, Andrew Norman, Theresa O’Connor, Kentaro Ohkouchi, Christian Ohler, Kenichi Okada, Alexandre Oliva, Bob Olson, Michael Olson, Takaaki Ota, Mark Oteiza, Pieter E. J. Pareit, Ross Patterson, David Pearson, Juan Pechiar, Jeff Peck, Damon Anton Permezel, Tom Perrine, William M. Perry, Per Persson, Jens Petersen, Nicolas Petton, Daniel Pfeiffer, Justus Piater, Richard L. Pieri, Fred Pierresteguy, Fran¸cois Pinard, Daniel Pittman, Christian Plaunt, Alexander Pohoyda, David Ponce, Noam Postavsky, Francesco A. Potort`ı, Michael D. Prange, Mukesh Prasad, Ken Raeburn, Marko Rahamaa, Ashwin Ram, Eric S. Raymond, Paul Reilly, Edward M. Reingold, David Reitter, Alex Rezinsky, Rob Riepel, Lara Rios, Adrian Robert, Nick Roberts, Roland B. Roberts, John Robinson, Denis B. Roegel, Danny Roozendaal, Sebastian Rose, William Rosenblatt, Markus Rost, Guillermo J. Rozas, Martin Rudalics, Ivar Rummelhoff, Jason Rumney, Wolfgang Rupprecht, Benjamin Rutt, Kevin Ryde, James B. Salem, Masahiko Sato, Timo Savola, Jorgen Sch¨afer, Holger Schauer, William Schelter, Ralph Schleicher, Gregor Schmid, Michael Schmidt, Ronald S. Schnell, Philippe Schnoebelen, Jan Schormann, Alex Schroeder, Stefan Schoef, Rainer Sch¨opf, Raymond Scholz, Eric Schulte, Andreas Schwab, Randal Schwartz, Oliver Seidel, Manuel Serrano, Paul Sexton, Hovav Shacham, Stanislav Shalunov, Marc Shapiro, Richard Sharman, Olin Shivers, Tibor Simko, Espen Skoglund, Rick Sladkey, Lynn Slater, Chris Smith, David ˇ Smith, Paul D. Smith, Wilson Snyder, William Sommerfeld, Simon South, Andre Spiegel, Michael Staats, Thomas Steffen, Ulf Stegemann, Reiner Steib, Sam Steingold, Ake Stenhoff, Philipp Stephani, Peter Stephenson, Ken Stevens, Andy Stewart, Jonathan Stigelman, Martin Stjernholm, Kim F. Storm, Steve Strassmann, Christopher Suckling, Olaf Sylvester, Naoto Takahashi, Steven Tamm, Jan Tatarik, Luc Teirlinck, Jean-Philippe Theberge, Jens T. Berger Thielemann, Spencer Thomas, Jim Thompson, Toru Tomabechi, David O’Toole, Markus Triska, Tom Tromey, Enami Tsugutomo, Eli Tziperman, Daiki Ueno, Masanobu Umeda, Rajesh Vaidheeswarran, Neil W. Van Dyke, Didier Verna, Joakim Verona, Ulrik Vieth, Geoffrey Voelker, Johan Vromans, Inge Wallin, John Paul Wallington, Colin Walters, Barry Warsaw, Christoph Wedler, Ilja Weis, Zhang Weize, Morten Welinder, Joseph Brian Wells, Rodney Whitby, John Wiegley, Sascha Wilde, Ed Wilkinson, Mike Williams, Roland Winkler, Bill Wohler, Steven A. Wood, Dale R. Worley, Francis J. Wright, Felix S. T. Wu, Tom Wurgler, Yamamoto Mitsuharu, Katsumi Yamaoka, Masatake Yamato, Jonathan Yavner, Ryan Yeske, Ilya Zakharevich, Milan Zamazal, Victor Zandy, Eli Zaretskii, Jamie Zawinski, Andrew Zhilin, Shenghuo Zhu, Piotr Zieli´nski, Ian T. Zimmermann, Reto Zimmermann, Neal Ziring, Teodor Zlatanov, and Detlev Zundel.