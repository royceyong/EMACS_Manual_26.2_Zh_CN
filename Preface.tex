\chapter{Preface:序言}
This manual documents the use and simple customization of the Emacs editor. Simple Emacs customizations do not  require you to be a programmer, but if you are not interested in customizing, you can ignore the customization hints.\par
这份手册记录了Emacs编辑器的用法和简单的自定义操作。简单的自定义操作不需要你是一个程序员,但如果你对自定义没有兴趣,你可以忽略关于自定义的章节。\par
This is primarily a reference manual, but can also be used as a primer. If you are new to Emacs, we recommend you start with the integrated, learn-by-doing tutorial, before reading the manual. To run the tutorial, start Emacs and type C-h t. The tutorial describes commands, tells you when to try them, and explains the results. The tutorial is available in several languages.\par
这是一份初级的参考手册,但也可以被当成进阶教材使用。如果你对于GNU DEmacs是新手,我们建议你阅读整合的、边做边学的指南。如果你需要查看指南,打开GNU Emacs键入“C-h t\footnote{译注:这个命令的意思是按住Ctrl键的同时按下“h”,再松开这两个键并按下“f”。}”这份指南介绍了常用的命令,告诉你什么时候尝试它们并告诉你结果。这份指南拥有多个语言\footnote{包括简体和繁体中文。请注意如果您使用的是GNU/Linux、BSD或者其他类UNIX操作系统相关发行版的终端模式,中文字符可能无法正确显示。}。\par
On first reading, just skim chapters 1 and 2, which describe the notational conventions of the manual and the general appearance of the Emacs display screen. Note which questions are answered in these chapters, so you can refer back later. After reading chapter 4, you should practice the commands shown there. The next few chapters describe fundamental techniques and concepts that are used constantly. You need to understand them thoroughly, so experiment with them until you are fluent.\par
在你第一次阅读时,请有选择地阅读第一与第二章,这两章介绍了此手册的常规以及Emacs屏幕的常规显示。请注意这两章所回答的问题以便日后查找。在阅读第四章后,请尝试那里的命令。之后几章介绍了最常用的基础的技术和概念。你需要完全理解它们,经常做实验直到十分熟练。\par
Chapters 14 through 19 describe intermediate-level features that are useful for many kinds of editing. Chapter 20 and following chapters describe optional but useful features; read those chapters when you need them.\par
第14到19章介绍了对于多种编辑有用的中间层面的功能。第20章及其以后介绍了可选但是有用的功能;按你所需阅读这些章节。\par
Read the Common Problems chapter if Emacs does not seem to be working properly. It explains how to cope with several common problems (see Section 34.2 [Dealing with Emacs Trouble], page 478), as well as when and how to report Emacs bugs (see Section 34.3 [Bugs], page 482).\par
如果Emacs似乎没有正常工作,阅读“常见问题”章。它解释了如何面对一些常见的问题(参见34.2【处理Emacs问题】),以及何时怎么上报Emacs的问题(参见43.3【问题】)。\par
To find the documentation of a particular command, look in the index. Keys (character commands) and command names have separate indexes. There is also a glossary, with a cross reference for each term.\par
如果你需要查找对于特定命令的支持,参见索引。键(键入命令的字符)与命令名字有不同的索引。还有一个名词索引以查找所有名词。\par
This manual is available as a printed book and also as an Info file. The Info file is for reading from Emacs itself, or with the Info program. Info is the principal format for documentation in the GNU system. The Info file and the printed book contain substantially the same text and are generated from the same source files, which are also distributed with GNU Emacs.\par
这份技术手册同时拥有纸质版和Info文件\footnote{译注:Info是GNU/Linux中的一种帮助系统。}。Info文件可以在Emacs中被阅读或者使用Info程序。Info文件与纸质书由相同的代码生成(同样随GNU Emacs分发)并且包含完全相同的内容。\par
GNU Emacs is a member of the Emacs editor family. There are many Emacs editors, all sharing common principles of organization. For information on the underlying philosophy of Emacs and the lessons learned from its development, see \textit{Emacs, the Extensible, Customizable Self-Documenting Display Editor}, available from \url{http://hdl.handle.net/1721.1/5736}.\par
GNU Emacs是Emacs编辑器家族的一员。目前有许多Emacs编辑器,它们共用相同的组织准则。对于有关Emacs现行的哲学以及从Emacs发展中得到的教训,参见\url{http://hdl.handle.net/1721.1/5736}的文章《Emacs, the Extensible, Customizable Self-Documenting Display Editor》。\par
This version of the manual is mainly intended for use with GNU Emacs installed on GNU and Unix systems. GNU Emacs can also be used on MS-DOS, Microsoft Windows, and Macintosh systems. The Info file version of this manual contains some more information about using Emacs on those systems. Those systems use different file name syntax; in addition MS-DOS does not support all GNU Emacs features. See Appendix G [Microsoft Windows], page 542, for information about using Emacs on Windows. See Appendix F [Mac OS / GNUstep], page 539, for information about using Emacs on Macintosh (and
GNUstep).\par
这个版本的手册主要是为了安装在GNU与UNIX操作系统上的GNU Emacs设计的,GNU Emacs也在MS-DOS、微软视窗操作系统与麦金塔机上使用。这份手册的Info文件版本包括这些系统上的更多信息。这些操作系统使用不同的文件名;并且MS-DOS也不完全支持GNU Emacs的功能。参见附录【微软视窗操作系统】以查找视窗操作系统上使用的GNU Emacs的相关信息。参见附录【MacOS/GNUstep】以查找在麦金塔机上使用GNU Emacs的信息。