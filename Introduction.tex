%%%%%%%%%%%%%%%%%%%%%%%%%%%%%%%%%%%%%%%%%%
%Copyright (C) 2018-2020  YuZJLab
%This program is free software: you can redistribute it and/or modify
%it under the terms of the GNU General Public License as published by
%the Free Software Foundation, either version 3 of the License, or
%(at your option) any later version.
%This program is distributed in the hope that it will be useful,
%but WITHOUT ANY WARRANTY; without even the implied warranty of
%MERCHANTABILITY or FITNESS FOR A PARTICULAR PURPOSE.  See the
%GNU General Public License for more details.
%You should have received a copy of the GNU General Public License
%along with this program.  If not, see <https://www.gnu.org/licenses/>.
%%%%%%%%%%%%%%%%%%%%%%%%%%%%%%%%%%%%%%%%%%
\chapter{Introduction:对于GNU Emacs的介绍}
You are reading about GNU Emacs, the GNU incarnation of the advanced, self-documenting, customizable, extensible editor Emacs. (The ‘G’ in GNU (GNU’s Not Unix) is not silent.)\par
你正在阅读GNU Emacs,高级的、具有完全帮助的、可完全自定义的、可扩展的编辑器Emacs的GNU版本(GNU(GNU's Not Unix)中的“G”不是不发音的)。\par
We call Emacs \textit{advanced} because it can do much more than simple insertion and deletion of text. It can control subprocesses, indent programs automatically, show multiple files at once, and more. Emacs editing commands operate in terms of characters, words, lines, sentences, paragraphs, and pages, as well as expressions and comments in various programming languages.\par
我们称Emacs“高级的”因为它能够做除了简单的插入和删除以外的很多事。它能够控制子进程,同时展示多个文件等等。Emacs的编辑命令是靠键入字符、单词、句子、段落和页面以及各种编程语言的表达和注释完成的。\par
\textit{Self-documenting} means that at any time you can use special commands, known as help commands, to find out what your options are, or to find out what any command does, or to find all the commands that pertain to a given topic. See Chapter 7 [Help], page 39.\par
“具有完全帮助的”意为任何时候你可以使用称为“帮助命令”的特殊命令来明白你的操作是什么,或者其它命令将会干什么,或者找到关于特定主体的所有命令。参见第七章【帮助】。\par
\textit{Customizable} means that you can easily alter the behavior of Emacs commands in simple ways. For instance, if you use a programming language in which comments start with ‘<**’ and end with ‘**>’, you can tell the Emacs comment manipulation commands to use those strings (see Section 23.5 [Comments], page 268). To take another example, you can rebind the basic cursor motion commands (up, down, left and right) to any keys on the keyboard that you find comfortable. See Chapter 33 [Customization], page 444.\par
“可完全自定义的”意为你能够易如反掌地改变Emacs命令的用法。比如说,如果你使用一种注释由“<**”开头,由“**>”结尾的编程语言,你可以告诉Emacs的注释操作命令使用这些字符串(参见23.5节【注释】)。再举个例子:你可以将基本光标移动命令(上、下、左、右)重新关联到你觉得舒服的不同键。参见第33章【自定义】。\par
\textit{Extensible} means that you can go beyond simple customization and create entirely new commands. New commands are simply programs written in the Lisp language, which are run by Emacs’s own Lisp interpreter. Existing commands can even be redefined in the middle of an editing session, without having to restart Emacs. Most of the editing commands in Emacs are written in Lisp; the few exceptions could have been written in Lisp but use C instead for efficiency. Writing an extension is programming, but non-programmers can use it afterwards. See Section “Preface” in \textit{An Introduction to Programming in Emacs Lisp}, if you want to learn Emacs Lisp programming.\par
“可扩展的”意为你能够越过简单的自定义创建自己的新命令。新命令应是由Lisp语言书写的,由Emacs自己的Lisp解释器执行的程序。甚至连“退出”指令都能够在编辑时被重新定义。大多数编辑命令都是有Lisp书写的;特殊的排除项虽然也能够使用Lisp书写,但为了更强大的效率使用C语言。写扩展是编程,但是非编程人员可以在之后使用它。如果你想要学习Lisp编程,请参见《An Introduction to Programming in Emacs Lisp》的“Preface”章节。