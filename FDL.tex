\chapter{GNU Free Documentation License}
Version 1.3, 3 November 2008\par
Copyright © 2000, 2001, 2002, 2007, 2008 Free Software Foundation, Inc. <\url{https://fsf.org/}> \par
Everyone is permitted to copy and distribute verbatim copies of this license document, but changing it is not allowed.
\section{0. PREAMBLE}
The purpose of this License is to make a manual, textbook, or other functional and useful document "free" in the sense of freedom: to assure everyone the effective freedom to copy and redistribute it, with or without modifying it, either commercially or noncommercially. Secondarily, this License preserves for the author and publisher a way to get credit for their work, while not being considered responsible for modifications made by others.\par
This License is a kind of "copyleft", which means that derivative works of the document must themselves be free in the same sense. It complements the GNU General Public License, which is a copyleft license designed for free software.\par
We have designed this License in order to use it for manuals for free software, because free software needs free documentation: a free program should come with manuals providing the same freedoms that the software does. But this License is not limited to software manuals; it can be used for any textual work, regardless of subject matter or whether it is published as a printed book. We recommend this License principally for works whose purpose is instruction or reference.
\section{1. APPLICABILITY AND DEFINITIONS}
This License applies to any manual or other work, in any medium, that contains a notice placed by the copyright holder saying it can be distributed under the terms of this License. Such a notice grants a world-wide, royalty-free license, unlimited in duration, to use that work under the conditions stated herein. The "Document", below, refers to any such manual or work. Any member of the public is a licensee, and is addressed as "you". You accept the license if you copy, modify or distribute the work in a way requiring permission under copyright law.\par
A "Modified Version" of the Document means any work containing the Document or a portion of it, either copied verbatim, or with modifications and/or translated into another language.\par
A "Secondary Section" is a named appendix or a front-matter section of the Document that deals exclusively with the relationship of the publishers or authors of the Document to the Document's overall subject (or to related matters) and contains nothing that could fall directly within that overall subject. (Thus, if the Document is in part a textbook of mathematics, a Secondary Section may not explain any mathematics.) The relationship could be a matter of historical connection with the subject or with related matters, or of legal, commercial, philosophical, ethical or political position regarding them.\par
The "Invariant Sections" are certain Secondary Sections whose titles are designated, as being those of Invariant Sections, in the notice that says that the Document is released under this License. If a section does not fit the above definition of Secondary then it is not allowed to be designated as Invariant. The Document may contain zero Invariant Sections. If the Document does not identify any Invariant Sections then there are none.\par
The "Cover Texts" are certain short passages of text that are listed, as Front-Cover Texts or Back-Cover Texts, in the notice that says that the Document is released under this License. A Front-Cover Text may be at most 5 words, and a Back-Cover Text may be at most 25 words.\par
A "Transparent" copy of the Document means a machine-readable copy, represented in a format whose specification is available to the general public, that is suitable for revising the document straightforwardly with generic text editors or (for images composed of pixels) generic paint programs or (for drawings) some widely available drawing editor, and that is suitable for input to text formatters or for automatic translation to a variety of formats suitable for input to text formatters. A copy made in an otherwise Transparent file format whose markup, or absence of markup, has been arranged to thwart or discourage subsequent modification by readers is not Transparent. An image format is not Transparent if used for any substantial amount of text. A copy that is not "Transparent" is called "Opaque".\par
Examples of suitable formats for Transparent copies include plain ASCII without markup, Texinfo input format, LaTeX input format, SGML or XML using a publicly available DTD, and standard-conforming simple HTML, PostScript or PDF designed for human modification. Examples of transparent image formats include PNG, XCF and JPG. Opaque formats include proprietary formats that can be read and edited only by proprietary word processors, SGML or XML for which the DTD and/or processing tools are not generally available, and the machine-generated HTML, PostScript or PDF produced by some word processors for output purposes only.\par
The "Title Page" means, for a printed book, the title page itself, plus such following pages as are needed to hold, legibly, the material this License requires to appear in the title page. For works in formats which do not have any title page as such, "Title Page" means the text near the most prominent appearance of the work's title, preceding the beginning of the body of the text.\par
The "publisher" means any person or entity that distributes copies of the Document to the public.\par
A section "Entitled XYZ" means a named subunit of the Document whose title either is precisely XYZ or contains XYZ in parentheses following text that translates XYZ in another language. (Here XYZ stands for a specific section name mentioned below, such as "Acknowledgements", "Dedications", "Endorsements", or "History".) To "Preserve the Title" of such a section when you modify the Document means that it remains a section "Entitled XYZ" according to this definition.\par
The Document may include Warranty Disclaimers next to the notice which states that this License applies to the Document. These Warranty Disclaimers are considered to be included by reference in this License, but only as regards disclaiming warranties: any other implication that these Warranty Disclaimers may have is void and has no effect on the meaning of this License.
\section{2. VERBATIM COPYING}
You may copy and distribute the Document in any medium, either commercially or noncommercially, provided that this License, the copyright notices, and the license notice saying this License applies to the Document are reproduced in all copies, and that you add no other conditions whatsoever to those of this License. You may not use technical measures to obstruct or control the reading or further copying of the copies you make or distribute. However, you may accept compensation in exchange for copies. If you distribute a large enough number of copies you must also follow the conditions in section 3.\par
You may also lend copies, under the same conditions stated above, and you may publicly display copies.
\section{3. COPYING IN QUANTITY}
If you publish printed copies (or copies in media that commonly have printed covers) of the Document, numbering more than 100, and the Document's license notice requires Cover Texts, you must enclose the copies in covers that carry, clearly and legibly, all these Cover Texts: Front-Cover Texts on the front cover, and Back-Cover Texts on the back cover. Both covers must also clearly and legibly identify you as the publisher of these copies. The front cover must present the full title with all words of the title equally prominent and visible. You may add other material on the covers in addition. Copying with changes limited to the covers, as long as they preserve the title of the Document and satisfy these conditions, can be treated as verbatim copying in other respects.\par
If the required texts for either cover are too voluminous to fit legibly, you should put the first ones listed (as many as fit reasonably) on the actual cover, and continue the rest onto adjacent pages.\par
If you publish or distribute Opaque copies of the Document numbering more than 100, you must either include a machine-readable Transparent copy along with each Opaque copy, or state in or with each Opaque copy a computer-network location from which the general network-using public has access to download using public-standard network protocols a complete Transparent copy of the Document, free of added material. If you use the latter option, you must take reasonably prudent steps, when you begin distribution of Opaque copies in quantity, to ensure that this Transparent copy will remain thus accessible at the stated location until at least one year after the last time you distribute an Opaque copy (directly or through your agents or retailers) of that edition to the public.\par
It is requested, but not required, that you contact the authors of the Document well before redistributing any large number of copies, to give them a chance to provide you with an updated version of the Document.
\section{4. MODIFICATIONS}
You may copy and distribute a Modified Version of the Document under the conditions of sections 2 and 3 above, provided that you release the Modified Version under precisely this License, with the Modified Version filling the role of the Document, thus licensing distribution and modification of the Modified Version to whoever possesses a copy of it. In addition, you must do these things in the Modified Version:\par
A. Use in the Title Page (and on the covers, if any) a title distinct from that of the Document, and from those of previous versions (which should, if there were any, be listed in the History section of the Document). You may use the same title as a previous version if the original publisher of that version gives permission. \par
B. List on the Title Page, as authors, one or more persons or entities responsible for authorship of the modifications in the Modified Version, together with at least five of the principal authors of the Document (all of its principal authors, if it has fewer than five), unless they release you from this requirement. \par
C. State on the Title page the name of the publisher of the Modified Version, as the publisher. \par
D. Preserve all the copyright notices of the Document. \par
E. Add an appropriate copyright notice for your modifications adjacent to the other copyright notices. \par
F. Include, immediately after the copyright notices, a license notice giving the public permission to use the Modified Version under the terms of this License, in the form shown in the Addendum below. \par
G. Preserve in that license notice the full lists of Invariant Sections and required Cover Texts given in the Document's license notice. \par
H. Include an unaltered copy of this License. \par
I. Preserve the section Entitled "History", Preserve its Title, and add to it an item stating at least the title, year, new authors, and publisher of the Modified Version as given on the Title Page. If there is no section Entitled "History" in the Document, create one stating the title, year, authors, and publisher of the Document as given on its Title Page, then add an item describing the Modified Version as stated in the previous sentence. \par
J. Preserve the network location, if any, given in the Document for public access to a Transparent copy of the Document, and likewise the network locations given in the Document for previous versions it was based on. These may be placed in the "History" section. You may omit a network location for a work that was published at least four years before the Document itself, or if the original publisher of the version it refers to gives permission. \par
K. For any section Entitled "Acknowledgements" or "Dedications", Preserve the Title of the section, and preserve in the section all the substance and tone of each of the contributor acknowledgements and/or dedications given therein. \par
L. Preserve all the Invariant Sections of the Document, unaltered in their text and in their titles. Section numbers or the equivalent are not considered part of the section titles. \par
M. Delete any section Entitled "Endorsements". Such a section may not be included in the Modified Version. \par
N. Do not retitle any existing section to be Entitled "Endorsements" or to conflict in title with any Invariant Section. 
O. Preserve any Warranty Disclaimers.\par
If the Modified Version includes new front-matter sections or appendices that qualify as Secondary Sections and contain no material copied from the Document, you may at your option designate some or all of these sections as invariant. To do this, add their titles to the list of Invariant Sections in the Modified Version's license notice. These titles must be distinct from any other section titles.\par
You may add a section Entitled "Endorsements", provided it contains nothing but endorsements of your Modified Version by various parties—for example, statements of peer review or that the text has been approved by an organization as the authoritative definition of a standard.\par
You may add a passage of up to five words as a Front-Cover Text, and a passage of up to 25 words as a Back-Cover Text, to the end of the list of Cover Texts in the Modified Version. Only one passage of Front-Cover Text and one of Back-Cover Text may be added by (or through arrangements made by) any one entity. If the Document already includes a cover text for the same cover, previously added by you or by arrangement made by the same entity you are acting on behalf of, you may not add another; but you may replace the old one, on explicit permission from the previous publisher that added the old one.\par
The author(s) and publisher(s) of the Document do not by this License give permission to use their names for publicity for or to assert or imply endorsement of any Modified Version.
\section{5. COMBINING DOCUMENTS}
You may combine the Document with other documents released under this License, under the terms defined in section 4 above for modified versions, provided that you include in the combination all of the Invariant Sections of all of the original documents, unmodified, and list them all as Invariant Sections of your combined work in its license notice, and that you preserve all their Warranty Disclaimers.\par
The combined work need only contain one copy of this License, and multiple identical Invariant Sections may be replaced with a single copy. If there are multiple Invariant Sections with the same name but different contents, make the title of each such section unique by adding at the end of it, in parentheses, the name of the original author or publisher of that section if known, or else a unique number. Make the same adjustment to the section titles in the list of Invariant Sections in the license notice of the combined work.\par
In the combination, you must combine any sections Entitled "History" in the various original documents, forming one section Entitled "History"; likewise combine any sections Entitled "Acknowledgements", and any sections Entitled "Dedications". You must delete all sections Entitled "Endorsements".
\section{6. COLLECTIONS OF DOCUMENTS}
You may make a collection consisting of the Document and other documents released under this License, and replace the individual copies of this License in the various documents with a single copy that is included in the collection, provided that you follow the rules of this License for verbatim copying of each of the documents in all other respects.\par
You may extract a single document from such a collection, and distribute it individually under this License, provided you insert a copy of this License into the extracted document, and follow this License in all other respects regarding verbatim copying of that document.
\section{7. AGGREGATION WITH INDEPENDENT WORKS}
A compilation of the Document or its derivatives with other separate and independent documents or works, in or on a volume of a storage or distribution medium, is called an "aggregate" if the copyright resulting from the compilation is not used to limit the legal rights of the compilation's users beyond what the individual works permit. When the Document is included in an aggregate, this License does not apply to the other works in the aggregate which are not themselves derivative works of the Document.\par
If the Cover Text requirement of section 3 is applicable to these copies of the Document, then if the Document is less than one half of the entire aggregate, the Document's Cover Texts may be placed on covers that bracket the Document within the aggregate, or the electronic equivalent of covers if the Document is in electronic form. Otherwise they must appear on printed covers that bracket the whole aggregate.
\section{8. TRANSLATION}
Translation is considered a kind of modification, so you may distribute translations of the Document under the terms of section 4. Replacing Invariant Sections with translations requires special permission from their copyright holders, but you may include translations of some or all Invariant Sections in addition to the original versions of these Invariant Sections. You may include a translation of this License, and all the license notices in the Document, and any Warranty Disclaimers, provided that you also include the original English version of this License and the original versions of those notices and disclaimers. In case of a disagreement between the translation and the original version of this License or a notice or disclaimer, the original version will prevail.\par
If a section in the Document is Entitled "Acknowledgements", "Dedications", or "History", the requirement (section 4) to Preserve its Title (section 1) will typically require changing the actual title.
\section{9. TERMINATION}
You may not copy, modify, sublicense, or distribute the Document except as expressly provided under this License. Any attempt otherwise to copy, modify, sublicense, or distribute it is void, and will automatically terminate your rights under this License.\par
However, if you cease all violation of this License, then your license from a particular copyright holder is reinstated (a) provisionally, unless and until the copyright holder explicitly and finally terminates your license, and (b) permanently, if the copyright holder fails to notify you of the violation by some reasonable means prior to 60 days after the cessation.\par
Moreover, your license from a particular copyright holder is reinstated permanently if the copyright holder notifies you of the violation by some reasonable means, this is the first time you have received notice of violation of this License (for any work) from that copyright holder, and you cure the violation prior to 30 days after your receipt of the notice.\par
Termination of your rights under this section does not terminate the licenses of parties who have received copies or rights from you under this License. If your rights have been terminated and not permanently reinstated, receipt of a copy of some or all of the same material does not give you any rights to use it.
\section{10. FUTURE REVISIONS OF THIS LICENSE}
The Free Software Foundation may publish new, revised versions of the GNU Free Documentation License from time to time. Such new versions will be similar in spirit to the present version, but may differ in detail to address new problems or concerns. See \url{https://www.gnu.org/licenses/}.\par
Each version of the License is given a distinguishing version number. If the Document specifies that a particular numbered version of this License "or any later version" applies to it, you have the option of following the terms and conditions either of that specified version or of any later version that has been published (not as a draft) by the Free Software Foundation. If the Document does not specify a version number of this License, you may choose any version ever published (not as a draft) by the Free Software Foundation. If the Document specifies that a proxy can decide which future versions of this License can be used, that proxy's public statement of acceptance of a version permanently authorizes you to choose that version for the Document.
\section{11. RELICENSING}
"Massive Multiauthor Collaboration Site" (or "MMC Site") means any World Wide Web server that publishes copyrightable works and also provides prominent facilities for anybody to edit those works. A public wiki that anybody can edit is an example of such a server. A "Massive Multiauthor Collaboration" (or "MMC") contained in the site means any set of copyrightable works thus published on the MMC site.\par
"CC-BY-SA" means the Creative Commons Attribution-Share Alike 3.0 license published by Creative Commons Corporation, a not-for-profit corporation with a principal place of business in San Francisco, California, as well as future copyleft versions of that license published by that same organization.\par
"Incorporate" means to publish or republish a Document, in whole or in part, as part of another Document.\par
An MMC is "eligible for relicensing" if it is licensed under this License, and if all works that were first published under this License somewhere other than this MMC, and subsequently incorporated in whole or in part into the MMC, (1) had no cover texts or invariant sections, and (2) were thus incorporated prior to November 1, 2008.\par
The operator of an MMC Site may republish an MMC contained in the site under CC-BY-SA on the same site at any time before August 1, 2009, provided the MMC is eligible for relicensing.\par
\section{ADDENDUM: How to use this License for your documents}
To use this License in a document you have written, include a copy of the License in the document and put the following copyright and license notices just after the title page:\par
\begin{verbatim}
Copyright (C)  YEAR  YOUR NAME.
Permission is granted to copy, distribute and/or modify this document
under the terms of the GNU Free Documentation License, Version 1.3
or any later version published by the Free Software Foundation;
with no Invariant Sections, no Front-Cover Texts, and no Back-Cover Texts.
A copy of the license is included in the section entitled "GNU
Free Documentation License".
\end{verbatim}    
If you have Invariant Sections, Front-Cover Texts and Back-Cover Texts, replace the "with … Texts." line with this:
\begin{verbatim}
with the Invariant Sections being LIST THEIR TITLES, with the
Front-Cover Texts being LIST, and with the Back-Cover Texts being LIST.
\end{verbatim}  
If you have Invariant Sections without Cover Texts, or some other combination of the three, merge those two alternatives to suit the situation.\par
If your document contains nontrivial examples of program code, we recommend releasing these examples in parallel under your choice of free software license, such as the GNU General Public License, to permit their use in free software. 

\chapter{GNU 自由文本授權 (繁體中文翻譯版)}
第1.3版, 2008年11月
\section{聲明!}
This is an unofficial translation of the GNU Free Document License into Chinese. It was not published by the Free Software Foundation, and does not legally state the distribution terms for software that uses the GNU FDL--only the original English text of the GNU FDL does that. However, we hope that this translation will help Chinese speakers understand the GNU FDL better. You may publish this translation, modified or unmodified, only under the terms at http://www.gnu.org/licenses/translations.html\par
這是GNU通用公共授權合約的一份非官方中文翻譯,並非自由軟體基金會所發表,不適用於使用GNU通用公共授權合約發佈的軟體的法律聲明——只有GNU通用公共授權合約英文原版才具有法律效力。不過我們希望本翻譯能夠幫助中文讀者更好地理解GNU通用公共授權合約。\par 您可以僅根據http://www.gnu.org/licenses/translations.html中的條款發布此修改或未修改的翻譯。\par
著作權所有 (C) 2000,2001,2002,2007,2008 Free Software Foundation, Inc.<\url{http://fsf.org/}>\par
允許每個人複製和發佈本授權文件的完整副本,\par
但不允許對它進行任何修改。
\section{0. 導言}
本授權的目的在於作爲一種手冊、教科書或其他的具有功能性的有用文件獲得自由:確保每一個人都具有複製和重新發佈它的有效自由,而不論是否作出修改,也不論其是否具有商業行爲。其次,本授權保存了作者以及出版者由於他們的工作而 得到 名譽的方式,同時也不被認爲應該對其他人所作出的修改而擔負責任。\par
本授權是一種’copyleft’,這表示文件的衍生作品本身必須具有相同的自由涵義。它補足了GNU 公共通用授權-- 一種爲了自由軟體而設計的’copyleft’授權。\par
我們設計了本授權是爲了將它使用到自由軟體的使用手冊上,因爲自由軟體需要自由的文檔:一種自由的程式應該提供與此軟體具有相同的自由的使用手冊。但是本授權並不被限制在軟體使用手冊的應用上;它可以被用於任何以文字作基礎的作品,而不論其主題內容,或者它是否是一個被出版的印刷書籍。我們建議本授權主要應用在以使用說明或提供參考作爲目的的作品上。
\section{1. 效力與定義}
本授權的效力在於任何媒體中的任何的使用手冊或其他作品,只要其中包含由版權所有人所指定的聲明,說明它可以在本授權的條款下被發佈。這樣的一份聲明提供了全球範圍內的,免版稅的和沒有期限的許可,在此所陳述條件下使用那個作品。以下所稱的文件,指的是任何像這樣的使用手冊或作品。公衆中的任何成員都是被許可人,並且稱作爲你。如果你以一種需要在版權法下取得允許的方式進行複製、修改或發佈作品,你就接受了這項許可。\par
“修改版本”指的是任何包含文件或是它的其中一部份,不論是逐字的複製或是經過修正,或翻譯成其他語言的任何作品。\par
“ 次要章節”是一個具名的附錄,或是文件的本文之前內容的章節,專門用來處理文件的出版者或作者,與文件整體主題(或其他相關內容)的關係,並且不包含任何可以直接落入那個整體主題的內容。(因此,如果文件的部分內容是作爲數學教科書,那麽其次要章節就可以不用來解釋任何數學。)它的關係可以是與主題相關的歷史連接,或是與其相關的法律、商業、哲學、倫理道德或政治立場。\par
“不變章節”是標題已被指定的某些次要章節,在一個聲明了是以本授權加以發行的文件中,依此作爲不變章節。如果一個章節並不符合上述有關於次要的定義時,則它並不允許被指定爲不變。文件可以不包含不變章節。如果文件並沒有指出任何不變章節,那麽就是沒有。\par
“封面文字”是某些被加以列出的簡短文字段落,在一個聲明了是以本授權加以發行的文件中,依此作爲前封面文字或後封面文字。前封面文字最多可以包含 5 個單詞,後封面文字最多可以包含 25 個單詞。\par
文件的”透明”拷貝指的是一份機器可讀的拷貝,它以一種一般公衆可以取得其規格說明的格式來表現,適合於直接用一般文字編輯器、一般點陣圖像程式用於由圖元圖元構成的影像或一些可以廣泛取得的繪圖程式用於由向量繪製的圖形直接地進行修訂;並且適合於輸入到文字格式化程式,或是可以自動地轉換到適合於輸入到文字格式化程式的各種格式。一份以透明以外的檔案格式所構成的拷貝,其標記或缺少標記,若是被安排成用來挫折或是打消讀者進行其後續的修改,則此拷貝並非透明。一種影像格式,如果僅僅是用來充斥文本的資料量時,就不是透明的。一個不是透明的拷貝被稱爲混濁。\par
透明拷貝適合格式的例子包括有:沒有標記的純 ASCII、Texinfo 輸入格式、LaTeX 輸入格式、使用可以公開取得其 DTD 的 SGML 或 XML、合乎標準的簡單 HTML、PostScript 或 PDF。透明影像格式的例子有 PNG、XCF 和 JPG 。混濁格式包括只能夠以私人文書處理器閱讀以及編輯的私人格式、DTD 以及或處理工具不能夠一般地加以取得的 SGML 或 XML、以及由某些文書處理器只是爲了輸出的目的而做出的,由機器製作的 HTML、PostScript 或 PDF 。\par
標題頁對一本印刷書籍來說,指的是標題頁本身,以及所需要用來容納本授權必須出現在標題頁的易讀內容的,如此的接續數頁。對於並沒有任何如此頁面的作品的某些格式,標題頁指的是本文主體開始之前作品標題最顯著位置的文字。\par
出版者指的是把那些文本副本發佈給公衆的任何人或實體。\par
一個標題爲 XYZ的章節指的是文件的一個具名的次要單元,其標題精確地爲 XYZ 或是將 XYZ 包含在跟著翻譯爲其他語言的 XYZ 文字後面的括弧內 -- 這裏 XYZ 代表的是名稱於下提及的特定章節,像是感謝、貢獻、背書或歷史。當你修改文件時,給像這樣子的章節保存其標題指的是,它保持爲一個根據這個定義的標題爲 XYZ的章節。\par
文件可以在用來陳述本授權效力及於文件的聲明後,包括擔保放棄。這些擔保放棄被考慮爲以提及的方式,包括在本授權中,但是只被看作爲放棄擔保之用:任何這些擔保放棄可能會有的其他暗示都是無效的,並且也對本授權的含義沒有影響。\par
以下是有關複製、發佈及修改的明確條款及條件。
\section{2. 逐字的複製}
你可以複製或發佈文件於任何媒體,而不論其是否具有商業買賣行爲,其條件爲具有本授權、版權聲明和許可聲明,說明本授權效力於文件的所有重制拷貝,並且你沒有增加任何其他條件到本授權的條件中。你不可以使用技術手段,來妨礙或控制你所製作或發佈的拷貝閱讀或進一步的發佈。然而,你可以接受補償以作爲拷貝的交換。如果你發佈了數量足夠大的拷貝,你也必須遵循第三條的條件。\par
你也可以在上述的相同條件下借出拷貝,並且你可以公開地陳列拷貝。
\section{3. 大量地複製}
如果你出版文件的印刷拷貝或者通常具有印刷封面的媒體的拷貝,數量上超過一百個單位,而且文件許可聲明要求有封面文字,那麽你必須將這些拷貝附上清楚且易讀的文字:前封面文字於前封面上、後封面文字於後封面上。這兩種封面必須清楚易讀地辨認出,你是這些拷貝的出版者。前封面文字必須展示完整的標題,而標題的文字應當同等地顯著可見。你可以增加額外的內容於封面上。僅在封面作出改變的複製,只要它們保存了文件的標題,並且滿足了這些條件,可以在其他方面被看作爲逐字的複製。\par
如果對於任意一個封面所需要的文字,數量過於龐大以至於不能符合易讀的原則,你應該在實際封面的最前面列出所能符合易讀原則的內容,然後將剩下的接續在相鄰的頁面。\par
如果你出版或發佈數量超過一百個單位文件的混濁拷貝,你必須與此混濁拷貝一起包含一份機器可讀的透明拷貝,或是與一份混濁拷貝一起或其陳述一個電腦網路位元址,使一般的網路使用公衆具有存取權,可以使用公開標準的網路協定,下載一份文件的完全透明拷貝,此拷貝中並且沒有增加額外的內容。如果你使用後面的選項,當你開始大量地發佈混濁拷貝時,你必須採取合理的審慎步驟,以保證這個透明拷貝將會在發佈的一開始就保持可供存取,直到你最後一次發佈那個發行版的一份混濁拷貝給公衆後,至少一年爲止。以保證這個透明拷貝,將會在所陳述的位址保持如此的可存取性,直到你最後一次發佈那個發行版直接或經由你的代理商或零售商的一份混濁拷貝給公衆後,至少一年爲止。\par
你被要求,但不是必須,在重新發佈任何大數量的拷貝之前與文件的作者聯絡,給予他們提供你一份文件的更新版本的機會。
\section{4. 修改}
你可以在上述第二條和第三條的條件下,複製和發佈文件的修改版本,其條件爲你要精確地在本授權下發佈修改版本,且修改版本補足了文件的角色,從而允許修改版本的發佈和修改權利給任何擁有它拷貝的人。另外,你必須在修改版本中做這些事:\par
第一款、 在標題頁或在封面上使用,如果有與先前版本不同的文件,應該被列在文件的歷史章節不同的標題。如果版本的原始出版者允許,你可以使用與某一個先前版本相同的標題。\par
第二款、 在修改版本的標題頁上列出擔負作者權的一個或多個人或實體作爲作者,並且列出至少五位文件的主要作者。如果少於五位元,則列出全部的主要作者,除非他們免除了你這個要求。\par
第三款、 在標題頁陳述修改版本的出版者的名稱作爲出版者。\par
第四款、 保存文件的所有版權聲明。\par
第五款、 爲你的修改增加一個與其他版權聲明相鄰的適當的版權聲明。\par
第六款、 在版權聲明後面,以授權附錄所顯示的形式,包括一個給予公衆在本授權條款下使用修改版本的許可聲明。\par
第七款、 在那個許可聲明中保存恒常章節和文件許可聲明中必要封面文字的全部列表。\par
第八款、 包括一個未被改變的本授權的副本。\par
第九款、 保存標題爲歷史的章節和其標題,並且增加一項至少陳述如同在標題頁中所給的修改版本的標題、年份、新作者和出版者。如果在文件中沒有標題爲歷史的章節,則製作出一個陳述如同在它的標題頁中所給的文件的標題、年份、新作者和出版者,然後增加一項描述修改版本如前面句子所陳述的情形。\par
第十款、 如果有的話,保存在文件中爲了給公衆存取文件的透明拷貝,而給予的網路位元址,以及同樣地在文件中爲了它所根據的先前版本,而給予的網路位元址。這些可以被放置在歷史章節。你可以省略一個在文件本身之前,已經至少出版了四年的作品的網路位元址,或是如果它所參照的那個版本的原始出版者給予允許的情形下也可以省略它。\par
第十一款、 在任何標題爲感謝或貢獻的章節,保存章節的標題,並且在那章節保存到那時候爲止,每一個貢獻者的感謝以及或貢獻的所有聲色。\par
第十二款、 保存文件的所有恒常章節,於其文字以及標題皆不得變更。章節號碼或其同等物並不被認爲是章節標題的一部份。\par
第十三款、 刪除任何標題爲背書的章節。這樣子的章節不可以被包括在修改版本中。\par
第十四款、 不要重新命名任何現存的章節,而使其標題爲背書,或造成與任何恒常章節相衝突的標題。\par
第十五款、 保存任何的擔保放棄。\par
如果修改版本包括新的本文之前內容的章節,或合乎作爲次要章節的附錄,並且沒有包含複製自文件的內容,則你具有選擇可以指定一些或全部這些章節爲恒常的。要這樣做,將它們的標題增加到在修改版本許可聲明中的恒常章節列表中。這些標題必須可以和任何其他章節標題加以區別。\par
你可以增加一個標題爲背書的章節,其條件爲它僅只包含由許多團體所提供的你的修改版本的背書 -- 舉例來說,同儕評審的說明,或本文已經被一個機構認可爲一個標準的權威定義。\par
你可以增加一個作爲前封面文字的最多五個字的段落,以及一個作爲後封面文字的最多二十五個字的段落,到修改版本的封面文字列表的後面。前封面文字和後封面文字都只能有一個段落,可以經由任何一個實體,或經由任何一個實體所作出的安排而被加入。如果文件已經在同樣的封面包括了封面文字 -- 先前由你或由你所代表的相同實體所作出的安排而加入,則你不可以增加另外一個;但是你可以在先前出版者的明確允許下替換掉舊的。
文件的作者和出版者並不由此授權,而給予允許使用他們的名字以爲了或經由聲稱或暗示任何修改版本背書爲自己所應得的方式而獲得名聲的權利。
\section{5. 組合文件}
你可以在上述第四條的條款中對於修改版本的定義之下,將文件與其他在本授權下發行的文件組合起來,其條件是你要在組合品中,包括所有原始文件的所有恒常章節,不做修改,同時在組合作品的許可聲明中將它們全部列爲恒常章節,並且你要保存它們所有的擔保放棄。\par
組合作品只需包含本授權的一份副本,並且重復的恒常章節可以僅以單一個拷貝來取代。如果名稱重復但內容不同的恒常章節,則將任此章節的標題,以在它的後面增加的方式加以獨特化,如果已知的話,於括弧中指出那個章節的原始作者或出版者的名稱,或是指定一個獨特的號碼。在此組合作品許可聲明中恒常章節的列表中,對其章節標題也作出相同的調整。\par
在組合品中,你必須組合在不同原始文件中,標題爲歷史的任何章節,形成一個標題爲歷史的章節;同樣組合任意標題爲感謝或貢獻的章節。你必須刪除標題爲背書的所有章節。
\section{6. 文件的收集}
你可以製作含有文件以及其他以本授權發行文件的收集品,並且將本授權對不同文件中的個別副本,以單一個包括在收集品的副本取代,其條件是你要遵循在其他方面,給予一個文件逐字複製的允許本授權的規則。\par
你可以從這樣的一個收集品中抽取出一份單一的文件,並且在本授權下將它單獨地發佈,其條件是你要在抽取出的文件中插入本授權的一份副本,並且在關於那份文件的逐字的複製的所有其他方面,遵循本授權。
\section{7. 獨立作品的聚集}
一個文件的編輯物,其中或附加於儲存物或發佈媒體的一冊的,具有其他分別且獨立的文件或作品的衍生品,如果經由編輯而産生的版權,並沒有用來限制此編輯物使用者的法律權力,而超過了個別的作品所允許的,則被稱爲一個聚集品。當文件中包括一個聚集品,本授權的效力並不僅在於此聚集品中的,於其本身並非文件的衍生作品的其他作品。\par
如果第三條的封面文字要求效力於這些文件的拷貝,並且文件的篇幅少於整個聚集品的一半,則文件的封面文字可以被放在只圍繞著文件,並於聚集品內部的封面或是電子的封面同等物上,如果文件是以電子的形式出現的話。否則它們必須出現在繞著整個聚集品的印刷封面上。
\section{8. 翻譯}
翻譯被認爲一種修改,因此你可以在第四條的條款下發佈文件的翻譯。用翻譯更換恒常章節需要取得版權所有者的特別允許,但是你可以包括部份或所有恒常章節的翻譯,使其附加到這些恒常章節的原始版本之中。你可以包括本授權、文件中的所有許可聲明和任何的擔保放棄的翻譯,其條件爲你也必須包括本授權的原始英文版本,以及這些聲明與放棄的原始版本。如果發生翻譯與本授權、聲明或放棄的原始版本有任何的不同意時,將以原始版本爲准。\par
如果在文件中的章節被標題爲感謝、貢獻或歷史,則保存標題第一條的必要條件第四條,典型上將會需要去更動實際的標題。
\section{9. 終止}
你不可以複製、修改、在本授權下再設定額外條件的次授權或發佈文件,除非明白地表示是在本授權所規範的條件下進行。任何其他的複製、修改、在本授權下再設定額外條件的次授權、或發佈文件的意圖都是無效的,並且將會自動地終止你在本授權下所被保障的權利。\par
然而,如果你終止所有違反本授權的行爲,特定版權所有人會暫時恢復你的授權直到此版權所有者明確並最終地終止你的授權。或者特定版權所有人永久地恢復你的授權如果此版權所有人在停止違反授權後的60天內沒有通過合理的方式通知你違反授權。\par
此外,此版權所有者用一些合理的方式通知你違反了授權規定,也是你第一次從此版權所有者收到違反授權的通知,並且你在收到通知後的30天內終止了這種行爲,那麽此版權所有者會永久地恢復你的授權。\par
你的權利在此章節的終止並不代表終止得到你的拷貝和權利的當事人的授權。如果你的權利被終止而沒有被永久性地恢復,那麽你將沒有任何權利去使用此章節的全部或部分資料和資料的拷貝。
\section{10. 本授權的未來改版}
自由軟體基金會可能偶爾會出版自由文件授權的新修訂過的版本。這種新版本在精神上將會類似于現在的版本,但在細節上可能會有不同,以對應新的問題或相關的事。請見 \url{http://www.gnu.org/copyleft/}。\par
本授權的任何版本都被指定一個可供區別的版本號碼。如果文件指定一個效力於它的特定號碼版本的本授權或任何以後的版本,你就具有選擇遵循指定的版本,或任何已經由自由軟體基金會出版的後來版本並且不是草稿的條款和條件。如果文件並沒有指定一個本授權的版本號碼,你就可以選擇任何一個曾由自由軟體基金會所出版不是草稿的版本。如果文件指定一個代理能夠決定本授權未來的那一種版本可以用,而且還指定代理公開聲明如果你接受一種版本你將會永久地被授權爲文本選擇此版本的權利。
\section{11.重新授權}
“MMC 網站”是指任何發佈有著作版權作品的網站伺服器,也爲任何人提供卓越的設施去編輯一些作品。任何人都可編輯的一種公衆維基就是這種伺服器的一個例子。包含在這個網站的”MMC”是指任何一套在MMC網站上發佈的具有著作版權的作品。\par
“CC-BY-SA”是指 Creative Commons Attribution-Share Alike 3.0 授權,它是被“知識共用組織” 頒佈的,“知識共用組織” 是一家非贏利性的,在聖弗朗西斯科,加利福尼亞具有重要的商業地位的組織。 而且未來的“copyleft”版本的授權也是被同一個組織發佈的。\par
“合併”是指以整體或作爲另一個文本的部分發佈或重新發佈一個文本。\par
MMC有重新授權的資格,如果它是在MMC下授權;或者所有的作品首次發佈並非在MMC下授權,後來以整體或部分合併到MMC下,它們沒有覆蓋性文本或不變的章節並且在2008年11月1日之前合併的。\par
那麽MMC網站的操作者會在2009年8月1日之前的任何時間在同一網站重新發佈包含在這一網站的MMC是經過CC-BY-SA 授權的,只要那個MMC有資格重新授權。
\section{授權附錄:如何使用本授權用於你的文件}
爲使用本授權成爲你撰寫成的一份文件,必須在文件中包括本授權的一份複本,以及標題頁的後面包括許可聲明:
\begin{verbatim}
Copyright (c) YEAR YOUR NAME. Permission is granted to copy, 
distribute and/or modify this document under the terms of the 
GNU Free Documentation License, Version 1.2 or any later version 
published by the Free Software Foundation; with no Invariant 
Sections, no Front-Cover Texts, and no Back-Cover Texts. A copy 
of the license is included in the section entitled "GNU Free 
Documentation License".
\end{verbatim}
如果你有恒常章節、前封面文字和後封面文字,請將with... Texts這一行以這些文本取代:
\begin{verbatim}
with the Invariant Sections being LIST THEIR TITLES, with the 
Front-Cover Texts being LIST, and with the Back-Cover Texts being LIST.
\end{verbatim}
如果你有不具封面文字的恒常章節,或一些其他這三者的組合,將可選擇的二項合併以符合實際情形。\par
如果你的文件中包含有並非微不足道的程式碼範例,我們建議這些範例平行地在你的自由軟體授權選擇下,比如以 GNU General Public License 的自由軟體授權來發佈,從而允許它們作爲自由軟體而使用。
\section{註腳}
原文網站: http://www.gnu.org/licenses/fdl.html\par
聯絡我們:Samuel Chong<schong2@go.pasadena.edu>