\chapter{屏幕的组织结构}
在一个图形界面(例如,使用X Windows系统的GNU/Linux操作系统)中,Emacs占用了一个图形窗口(Window)。在一个文本终端中,Emacs占用了一整个终端屏幕。我们将会使用“窗体”(Frame)表示Emacs占用的图形窗体或终端屏幕。Emacs在这两种窗体下表现十分相似\footnote{译注:虽然我不这么认为}。一般情况下Emacs在启动时创建一个窗体,但是你可以按照你的意愿创建另外的窗体(参见第18章【窗体】)。\par
Emacs窗体包括多个区分明显的区域。在窗体的最顶端是菜单栏(Menu Bar),它允许你通过一系列菜单访问命令。在一个图形界面上,菜单栏下方是工具栏(Tool Bar)。它由一系列图标构成,当你单击它是它将执行编辑命令。回显区(Echo Area)窗体的最底端,用于显示Emacs的提示信息或者当Emacs需要你输入时键入信息。\par
在一个Emacs窗体中,位于工具栏(如果存在的话)以及回显区的大块区域称为窗格(Window)。此后我们将会使用“窗格”来表示这个意思。图形界面系统一般会使用“窗口”(Window)来表示不同的意思;但是,正如已经声明的一样,我们将称这些图形界面窗口为“窗体”(Frame)。\par
缓冲区(Buffer)——文本、图像以及其他你正在编辑或浏览的东西所显示的地方——位于Emacs窗口。在一个图形界面上,窗格\footnote{译注:是“Window”。此后窗格一律称“Window”,窗体一律称“Frame”。窗口?那是什么东西?}拥有一个纵向的滚动条用于滚动显示整个缓冲区。窗格的最底端一栏(在终端界面下是一行文字)称为状态栏(Mode Line),用于显示关于缓冲区中正在进行的操作的的信息,比如说是否有未保存的改变,正在使用的编辑模式(Editing Mode),当前的行号等等。\par
通常情况下,当你启动Emacs的时候,一个窗体中只有一个窗格。但是,你可以将这个窗格水平或竖直地分为多个窗格,它们各自独立地显示一个缓冲区(参见第17章【窗格】)。\par
在任何时候,有且仅有一个窗格是活动窗格(Selected Window)。在一个图形界面下,活动窗格显示一个更加明显的光标(Crusor)(一般是闪动的黑色方块“$\blacksquare$”);其它窗格显示一个不明显的光标(一般是空心方块“$\square$”)。终端界面只将在活动窗格显示一个光标。在活动窗格显示的缓冲区称为活动缓冲区(“Selected Buffer”)并且就是正在被编辑的缓冲区。大部分Emacs命令含蓄地作用于活动缓冲区;在非活动的缓冲区中显示的文字仅仅为参考而显示。如果你使用多窗体的图形界面,选择一个窗体再选择一个窗格。